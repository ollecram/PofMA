\setcounter{chapter}{0}
\renewcommand{\thechapter}{1}
\chapter{The Real and Complex\\ Number Systems}\label{ch:1}
\setcounter{equation}{0}	        % To start with Equation 1
\counterwithin{equation}{chapter}	% Equation numbering will be 2.1 2.2 2.3   ... 

\section*{INTRODUCTION}\label{sec:1.1}

A satisfactory discussion of the main concepts of analysis (such as convergence, continuity, differentiation, and integration) must be based on an accurately defined number concept. We shall not, however, enter into any discussion of the axioms that govern the arithmetic of the integers, but assume familiarity with the rational numbers (i.e., the numbers of the form $m/n$, where $m$ and $n$ are integers and $n \neq 0$).

The rational number system is inadequate for many purposes, both as a field and as an ordered set. (these terms will be defined in Secs. 1.6 and 1.12.) For instance, there is no rational $p$ such that $p^2 = 2$. (we shall prove this presently.) This leads to the introduction of so-called \quotes{irrational numbers} which are often written as infinite decimal expansions and are considered to be \quotes{approximated} by the corresponding finite decimals. Thus the sequence
$$1, 1.4, 1.41, 1.414, 1.4142, \ldots$$ 
\quotes{tends to $\sqrt{2}.$} But unless the irrational number $\sqrt{2}$ has been clearly defined, the question must arise: Just what is it that this sequence \quotes{tends to}?

This sort of question can be answered as soon as the so-called \quotes{real number system} is constructed.
\subsection*{1.1 Example}
We now show that the equation
\begin{align*}
&(1) \quad\quad\quad\quad p^2 = 2
\end{align*}
is not satisfied by any rational $p$. If there were such a $p$, we could write $p = m/n$ where $m$ and $n$ are integers that are not both even. Let us assume this is done. Then (1) implies     
\begin{align*}
&(2) \quad\quad\quad\quad m^2 = 2n^2\,.
\end{align*}
This shows that $m^2$ is even. Hence $m$ is even (if $m$ would be odd, $m^2$ would be odd), and so $m^2$ is divisible by $4$. It follows that the right side of (2) is divisible by $4$, so that $n^2$ is even, which implies that $n$ is even.

The assumption that (1) holds thus leads to the conclusion that both $m$ and $n$ are even, contrary to our choice of $m$ and $n$. Hence (1) is impossible for rational $p$. 

We now examine this situation a little more closely. Let $A$ be the set of all positive rational $p$ such that $p^2 < 2$ and let $B$ consist of all positive rationals $p$ such that $p^2 > 2$. We shall show that $A$ \tit{contains no largest number and} $B$ \tit{contains no smallest.}

More explicitly, for every $p$ in $A$ we can find a rational $q$ in $A$ such that $p < q$, and for every $p$ in $B$ we can find a rational $q$ in $B$ such that $q < p$. 

To do this, we associate with each rational $p > 0$ the number 
\begin{align*}
&(3) \quad\quad\quad\quad q = p - \frac{p^2 - 2}{p + 2} = \frac{2p + 2}{p + 2}\,.
\end{align*}

Then 
\begin{align*}
&(4) \quad\quad\quad\quad q^2 - 2 = \frac{2 (p^2 - 2)}{(p + 2)^2}\,.
\end{align*}

If $p$ is in $A$ then $p^2 - 2 < 0$, (3) shows that $q > p$, and (4) shows that $q^2 < 2$. Thus $q$ is in $A$. 

If $p$ is in $B$ then $p^2 -2 > 0$, (3) shows that $0 < q < p$, and (4) shows that $q^2 > 2$. Thus $q$ is in $B$. 

\subsection*{1.2 Remark}
The purpose of the above discussion has been to show that the rational number system has certain gaps, in spite of the fact that between any two rationals there is another: If $r < s$ then $r < (r + s)/2 < s$. The real number system fills these gaps. This is the principal reason for the fundamental role which it plays in analysis. 

In order to elucidate its structure, as well as that of the complex numbers, we start with a brief discussion of the general concepts of \tit{ordered set} and \tit{field}.

Here is some of the standard set-theoretic terminology that will be used throughout this book. 

\subsection*{1.3 Definitions}
If $A$ is any set (whose elements may be numbers or any other objects), we write $x \in A$ to indicate that $x$ is a member (or an element) of $A$. 

If $x$ is not a member of $A$, we write $x \notin A$.

The set which contains no element will be called the \tit{empty set}. If a set has at least one element, it is called \tit{nonempty}.

If $A$ and $B$ are sets, and every element of $A$ is an element of $B$, we say that $A$ is a subset of $B$, and write $A \subset B$, or $B \supset A$. If, in addition, there is an element of $B$ which is not in $A$, then $A$ is said to be a \tit{proper} subset of $B$. Note that $A \subset A$ for every set $A$.

If $A \subset B$ and $B \subset A$, we write $A = B$. Otherwise, $A \neq B$.   

\subsection*{1.4 Definition}
Throughout Chap. 1, the set of all rational numbers will be denoted by $\Q$. 


\section*{ORDERED SETS}
\subsection*{1.5 Definition}
Let $S$ be a set. An \tit{order} on $S$ is a relation, denoted by $<\;$, with the following two properties:
\begin{enumerate}[(i)]
\item If $x \in S$ and $y \in S$, then one and only one of the statements
$$ x < y, \quad x = y, \quad y < x$$
is true. 
\item If $x, y, z \in S$, if $x < y$ and $y < z$, then $x < z$.
\end{enumerate}

The statement \quotes{$x < y$} may be read as \quotes{$x$ is less than $y$} or \quotes{$x$ is smaller than $y$} or \quotes{$x$ precedes $y$}.

It is often convenient to write $y > x$ in place of $x < y$.

The notation $x \leq y$ indicates that $x < y$ or $x = y$, without specifying which of these two is to hold. In other words, $x \leq y$ is the negation of $x > y$.

\subsection*{1.6 Definition}
An \tit{ordered set} is a set $S$ in which an order is defined.

For example, $\Q$ is an ordered set if $r < s$ is defined to mean that $s - r$ is a positive rational number.

\subsection*{1.7 Definition}
Suppose $S$ is an ordered set, and $E \subset S$. If there exists a $\beta \in S$ such that $x \leq \beta$ for every $x \in E$, we say that $E$ is \tit{bounded above}, and call $\beta$ an \tit{upper bound} of $E$.

Lower bounds are defined in the same way (with $\geq$ in place of $\leq$).  

\subsection*{1.8 Definition}
Suppose $S$ is an ordered set, $E \subset S$, and $E$ is bounded above. Suppose there exists an $\alpha \in S$ with the following properties:
\begin{enumerate}[(i)]
\item $\alpha$ is an upper bound of $E$. 
\item if $\gamma < \alpha$ then $\gamma$ is not an upper bound of $E$. 
\end{enumerate}

Then $\alpha$ is called the \tit{least upper bound} of $E$ [that there is at most one  such $\alpha$ is clear from (ii)] or the \tit{supremum} of $E$, and we write $$\alpha = \text{sup}\,E\,.$$

The \tit{greatest lower bound}, or \tit{infimum}, of a set $E$ which is bounded below is defined in the same manner: The statement $$\alpha = \text{inf}\,E$$
means that $\alpha$ is a lower bound of $E$ and that no $\beta$ with $\beta > \alpha$ is a lower bound of $E$. 
 
\subsection*{1.9 Examples}
\begin{enumerate}[(a)]
\item Consider the sets $A$ and $B$ of Example 1.1 as subsets of the ordered set $\Q$. The set $A$ is bounded above. In fact, the upper bounds of $A$ are exactly the members of $B$. Since $B$ contains no smallest member, $A$ \tit{has no least upper bound in} $\Q$.

Similarly, $B$ is bounded below: The set of all lower bounds of $B$ consists of $A$ and of all $r \in \Q$ with $r \leq 0$. Since $A$ has no largest member, $B$ \tit{has no greatest lower bound in} $\Q$.
   
\item If $\alpha = \text{sup}\,E$ exists, then $\alpha$ may or may not be a member of $E$. For instance, let $E_1$ be the set of all 
$r \in \Q$ with $r < 0$. Let $E_2$ be the set of all $r \in \Q$ with $r \leq 0$. Then 
$$\text{sup}\,E_1 = \text{sup}\,E_2 = 0\,,$$ 
and $0 \notin E_1,\; 0 \in E_2.$

\item Let $E$ consist of all numbers $1/n$, where $n = 1, 2, 3, \ldots$. Then $\text{sup}\,E = 1$, which is in $E$, 
and $\text{inf}\,E = 0$, which is not in $E$.  
\end{enumerate}

\subsection*{1.10 Definition}
An ordered set $S$ is said to have the \tit{least-upper-bound property} if the following is true:
\begin{list}{}{\leftmargin=\parindent\rightmargin=0pt}
\item If $E \subset S$, $E$ is not empty, and $E$ is bounded above, then $\text{sup}\,E$ exists in $S$.\\
Example 1.9(a) shows that $\Q$ does not have the least-upper-bound property. 
\end{list}
We shall now show that there is a close relation between greatest lower bounds and least upper bounds, and that every ordered set with the least-upper-bound property also has the greatest-lower-bound property.   

\subsection*{1.11 Theorem}
Suppose $S$ is an ordered set with the least-upper-bound property, $B \subset S$, $B$ is not empty, and $B$ is bounded below. Let $L$ be the set of all lower bounds of $B$. Then $$\alpha = \text{sup}\,L$$ exists in $S$ and $\alpha = \text{inf}\,B$. In particular $\text{inf}\,B$ exists in $S$.

\begin{list}{}{\leftmargin=\parindent\rightmargin=0pt}
\item \tbf{Proof  } Since $B$ is bounded below, $L$ is not empty. Since $L$ consists of exactly those $y \in S$ which satisfy the inequality $y \leq x$ for every $x \in B$, we see that \tit{every} $x \in B$ \tit{is an upper bound of} $L$. Thus $L$ is bounded above. 
Our hypothesis about $S$ implies therefore that $L$ has a supremum in $S$; call it $\alpha$.

If $\gamma < \alpha$  then (see Definition 1.8) $\gamma$ is not an upper bound of $L$, hence $\gamma \notin B$. It follows that 
$\alpha \leq x$ for every $x \in B$. Thus $\alpha \in L$. 

If $\alpha < \beta$ then $\beta \notin L$, since $\alpha$ is an upper bound of $L$. 

We have shown that $\alpha \in L$ but $\beta \notin L$ if $\beta > \alpha$. In other words, $\alpha$ is a lower bound of $B$, but $\beta$ is not if $\beta > \alpha$. This means that $\alpha = \text{inf}\,B$.       
\end{list}


\section*{FIELDS}

\subsection*{1.12 Definition}
A \tit{field} is a set $\F$ with two operations, called \tit{addition} and \tit{multiplication}, which satisfy the following so-called \quotes{field axioms} (A), (M), and (D):

\subsection*{(A)  Axioms for addition}
\begin{itemize}
\item[] (A1) If $x \in \F$ and $y \in F$, then their sum $x + y$ is in $\F$.
\item[] (A2) Addition is commutative: $x + y = y + x$ for all $x, y \in \F$. 
\item[] (A3) Addition is associative: $(x + y) + z = x + (y + z)$ for all $x, y, z \in \F$.
\item[] (A4) $\F$ contains an element $0$ such that $0 + x = x$ for every $x \in \F$. 
\item[] (A5) To every $x \in \F$ corresponds an element $-x \in \F$ such that $$x + (-x) = 0\,.$$
\end{itemize}

\subsection*{(M)  Axioms for multiplication}
\begin{itemize}
\item[] (M1) If $x \in \F$ and $y \in \F$, then their product $xy$ is in $\F$.
\item[] (M2) Multiplication is commutative: $xy = yx$ for all $x, y \in \F$.
\item[] (M3) Multiplication is associative: $(xy)z = x(yz)$ for all $x, y, z \in \F$.
\item[] (M4) $\F$ contains an element $1$ such that $1\,x = x$ for every $x \in \F$
\item[] (M5) If $x \in \F$ and $x \neq 0$ there exists an element $1/x\,\in \F$  such that $$x \cdot (1/x) = 1\,.$$
\end{itemize}

\subsection*{(M)  The distributive law}
$$x\,(y + z) = xy + xz$$ 
\begin{itemize}
\item[] holds for all $x, y, z\,\in \F$. 
\end{itemize}

\subsection*{1.13 Remarks}
\begin{itemize}
\item[] (a) One usually writes (in any field) 
$$x - y,\,\frac{x}{y},\, x + y + z,\, xyz,\, x^2,\, x^3,\, 2x,\, 3x,\, \ldots$$ in place of
$$x + (- y),\,x \cdot \frac{1}{y},\, (x + y) + z,\, (xy)z,\, xx,\, xxx,\, x + x,\, x + x + x,\, \ldots .$$

\item[] (b) The field axioms clearly hold in $\Q$, the set of all rational numbers, if addition and multiplication have their customary meaning. Thus $\Q$ is a field. 
\item[] (c) Although it is not our purpose to study fields (or any other algebraic structures) in detail, it is worthwhile to prove that some familiar properties are consequences of the field axioms; once we do this, we will not need to do it again for the real numbers and for the complex numbers. 
\end{itemize}

\subsection*{1.14 Proposition}
\tit{The axioms for addition imply the following statements.}
\begin{itemize}
\item[] (a) If $x + y = x + z$ \tit{then} $y = z\,.$   
\item[] (b) If $x + y = x$ \tit{then} $y = 0\,.$    
\item[] (c) If $x + y = 0$ \tit{then} $y = -x\,.$    
\item[] (d) $-(-x) = x\,.$
\item[] Statement (a) is a cancellation law. Note that (b) asserts the uniqueness of the element whose existence is assumed in (A4), and that (c) does the same for (A5).

\item[] \tbf{Proof } If $x + y = x + z$, the axioms (A) give 
\begin{align*}
y = 0 + y &= (- x + x) + y = -x + (x + y) \\
          &= -x + (x + z) = (-x + x) + z = 0 + z = z\,.
\end{align*}

\item[] This proves (a). Take $z = 0$ in (a) to obtain (b). Take $z = -x$ in (a) to obtain (c).
\item[] Since $-x + x =0,\,$ (c) (with $-x$ in place of $x$) gives (d).
\end{itemize}

\subsection*{1.15 Proposition}
\tit{The axioms for multiplication imply the following statements.}
\begin{itemize}
\item[] (a) If $x \neq 0$ and $xy = xz$ \tit{then} $ y = z\,.$   
\item[] (b) If $x \neq 0$ and $xy = x$  \tit{then} $ y = 1\,.$   
\item[] (c) If $x \neq 0$ and $xy = 1$  \tit{then} $ y = 1/x\,.$   
\item[] (d) If $x \neq 0$ $1/(1/x) = x\,.$   
\end{itemize}

The proof is so similar to that of Proposition 1.14 that we omit it.  

\subsection*{1.16 Proposition}
\tit{The field axioms imply the following statements, for any $x, y, z \in \F$.}
\begin{itemize}
\item[] (a) $0 x = 0$
\item[] (b) $x \neq 0$ and $y \neq 0$ \tit{then} $xy \neq 0\,.$
\item[] (c) $(-x) y = - (xy) = x(-y)\,.$
\item[] (d) $(-x)(-y) = xy\,.$
\item[] \tbf{Proof } $0x + 0x = (0 + 0) x = 0x\,.$ Hence 1.14(b) implies that $0x = 0$, and (a) holds. 
\item[] Next, assume $x \neq 0$, $y \neq 0$, but $xy = 0$. Then (a) gives
$$ 1 = \left(\frac{1}{y}\right) \left(\frac{1}{x}\right)\, xy = \left(\frac{1}{y}\right) \left(\frac{1}{x}\right)\, 0 = 0\,$$ 
a contradiction. Thus (b) holds.
\item[] The first equality in (c) comes from $$(-x)y + xy = (-x +x) y = 0 y = 0\,,$$ combined with 1.14(c); the other half of (c) is proved in the same way. Finally, 
$$(-x)(-y) = -[x(-y)] = -[-(xy)] = xy$$ by (c) and 1.14(d).
\end{itemize}

\subsection*{1.17 Definition}
\tit{An ordered field} is a field $\F$ which is also an \tit{ordered set}, such that
\begin{itemize}
\item[] (i) $x + y < x + z$ if $x, y, z \in \F$ and $y < z$,   
\item[] (ii) $xy > 0$ if $x \in \F$, $y \in \F$, $x > 0$ and $y > 0$. 
\end{itemize}

If $x > 0$ we call $x$ \tit{positive}; if $x < 0$, $x$ is \tit{negative}.

For example $\Q$ is an ordered field. 

All the familiar rules for working with inequalities apply in every ordered field: Multiplication by positive [negative]
quantities preserves [reverses] inequalities, no square is negative, etc. The following proposition lists some of these.

\subsection*{1.18 Proposition}
\tit{The following statements are true in every ordered field.}
\begin{itemize}
\item[] (a) \tit{If $x > 0$ then $-x < 0$, and vice versa.}
\item[] (b) \tit{If $x > 0$ and $y < z$ then $xy < xz$.} 
\item[] (c) \tit{If $x < 0$ and $y < z$ then $xy > xz$.}
\item[] (d) \tit{If $x \neq 0$ then $x^2 > 0$. In particular, $1 > 0$.}
\item[] (e) \tit{If $0 < x < y$ then $0 < 1/y < 1/x$.}
\item[] \tbf{Proof} 
\item[] (a) If $x > 0$ then $0 = -x + x > -x + 0$, so that $-x < 0$.\\
If $x < 0$ then $0 = -x + x < -x + 0$, so that $-x > 0$. This proves (a).   
\item[] (b) Since $z > y$, we have $z - y > y - y = 0$, hence $x(z - y) > 0$, and therefore 
$$xz = x (z - y) + xy > 0 + xy = xy\,.$$   
\item[] (c) By (a), (b) and Proposition 1.16(c),
$$-[x(z - y)] = (-x)(z - y) > 0\,,$$ so that $x(z - y) < 0$, hence $xz < xy\,.$  
\item[] (d) If $x > 0$, part (ii) of Definition 1.17 gives $x^2 > 0$. If $x < 0$, then $-x > 0$, hence $(-x)^2 > 0$. 
But $x^2 = (-x)^2$, by Proposition 1.16(d). Since $1 = 1^2$, $1 > 0$.     
\item[] (e) If $y > 0$ and $v \leq 0$, then $yv \leq 0$. But $y \cdot (1/y) =1 > 0$. Hence $1/y > 0$. 
Likewise, $1/x > 0$. If we multiply both sides of the inequality $x < y$ by the positive quantity 
$(1/x)(1/y)$, we obtain $(1/y) < (1/x)$.     
\end{itemize}


\section*{THE REAL FIELD}
We now state the \tit{existence theorem} which is the core of this chapter.


\subsection*{1.19 Theorem}
\tit{There exists an ordered field $\R$ which has the least-upper-bound property. Moreover, $\R$ contains $\Q$ as a subfield.}

The second statement means that $\Q \subset \R$ and that the operations of addition and multiplication in $\R$, when applied to members of $\Q$, coincide with the usual operations on rational numbers; also, the positive rational numbers are positive elements of $\R$.

The members of $\R$ are called \tit{real numbers}.

The proof of Theorem 1.1 is rather long and a bit tedious and is therefore presented in an Appendix to Chap. 1. The proof actually construct $\R$ from $\Q$. The next theorem could be extracted from this construction with very little extra effort. However, we prefer to derive it from Theorem 1.19 since this provides a good illustration of what one can do with the least-upper-bound property. 


\subsection*{1.20 Theorem}
\begin{itemize}
\item[] (a) \tit{If $x \in \R$, $y \in \R$, and $x > 0$, then there is a positive integer $n$ such that $$n x > y\,.$$}
\item[] (b) \tit{If $x \in \R$, $y \in \R$,  and $x < y$, then there is a $p \in \Q$ such that $x < p < y\,.$} 

\item[] Part (a) is usually referred to as the \tit{archimedean property} of $\R$. Part (b) may be stated saying that $\Q$ is \tit{dense} in $\R$: between any two real numbers there is a rational one.
\item[] \tbf{Proof }
\item[] (a) Let $A$ be the set of all $nx$, where $n$ runs through the positive integers. If (a) were false, then $y$ would be an upper bound of $A$. But then $A$ has a \tit{least} upper bound in $\R$. Put $\alpha = \text{sup} A$. Since $x > 0$, $\alpha - x < \alpha$, and 
$\alpha - x$ is not an upper bound of $A$. Hence $\alpha - x < mx$ for some positive integer $m$. But then $\alpha < (m + 1)x \in A$, which is impossible, since $\alpha$ is an upper bound of $A$.         
\item[] (b) Since $x < y$, we have $y - x > 0$, and (a) furnishes a positive integer $n$ such that $$n\,(y - x) > 1\,.$$
Apply (a) again, to obtain positive integers $m_1$ and $m_2$ such that $m_1 > nx,\:m_2 > -nx$. Then $$-m_2 < nx < m_1\,.$$ 
Hence there is an integer $m$ (with $-m_2 \leq m \leq m_1$) such that $$m - 1 \leq nx \leq m\,.$$ If we combine these inequalities, we obtain $$nx < m \leq 1 + nx < ny\,$$ Since $n > 0$, it follows that $$x < \frac{m}{n} < y\,.$$ This proves (b), with $p = m/n$. 
\end{itemize}

We shall now prove the existence of the $n$th roots of positive reals. This proof will show how the difficulty pointed out in the Introduction (irrationality of $\sqrt{2}$) can be handled in $\R$.

\subsection*{1.21 Theorem}
\tit{For every real $x > 0$ and every integer $n > 0$ there is one and only one positive real $y$ such that $y^n = x$.} 
\begin{itemize}
\item[] This number $y$ is written $\sqrt[n]{x}$ or $x^{1/n}$.
\item[]\tbf{Proof} That there is at most one such $y$ is clear, since $0 < y_1 < y_2$  implies $y_1^n < y_2^n$.

Let $E$ be the set consisting of all positive real numbers $t$ such that $t^n < x.$

If $t = x / (1 + x)$ then $0 \leq t < 1$. Hence $t^n \leq t < x$. Thus $t \in E$, and $E$ is nonempty.

if $t > 1 + x$ then $t^n \geq t > x$, so that $t \notin E$. Thus $1 + x$ is an upper bound of $E$. 

Hence Theorem 1.19 implies the existence of $$y = \text{sup} E\,.$$ 
To prove that $y^n = x$ we will show that each of the inequalities $y^n < x$ and $y^n > x$ leads to a contradiction.

The identity $b^n - a^n = (b-a)(b^{n-1} + b^{n-2}a + \cdots + a^{n-1})$ yields the inequality
$$ b^n - a^n < (b - a) n  b^{n-1}$$ when $0 < a < b$. 

Assume $y^n < x$. Choose $h$ so that $0 < h < 1$ and $$h < \frac{x - y^n}{n {(y + 1)}^{n - 1}}\;. $$
Put $a = y$, $b = y + h$. Then 
$${(y + h)}^n - y^n < hn {(y + h)}^{n - 1} < hn {(y + 1)}^{n - 1} < x - y^n\,.$$
Thus $({(y + h)}^n < x)\,,$ and $y + h \in E\,.$ Since $y + h > y\,,$ this contradicts the fact that $y$ is an upper bound of $E$. 

Assume $y^n > x$. Put  $$k = \frac{y^n - x}{n y^{n - 1}}\;. $$
Then $0 < k < y\,.$ If $t \geq y - k\,,$ we conclude that 
$$y^n - t^n \leq y^n - {(y - k)}^n < k n y^{n - 1} = y^n - x\,.$$ 
Thus $t^n > x$, and $t \notin E$. It follows that $y - k$ is an upper bound of $E$. 
But $y - k < y$, which contradicts the fact that $y$ is the \tit{least} upper bound of $E$.

Hence $y^n = x$, and the proof is complete.   
\end{itemize}

\subsection*{Corollary}
\tit{If $a$ and $b$ are positive real numbers and $n$ is a positive integer, then} $${(ab)}^{1/n} = a^{1/n} b^{1/n}\,.$$

\tbf{Proof}\:\:\: Put $\alpha = a^{1/n}\; \beta = b^{1/n}\;$. Then
$$ab = \alpha^n \beta^n = {(\alpha \beta)}^n\;,$$ 
since multiplication is commutative. [Axiom (M2) in Definition 1.12.]\\
The uniqueness assertion of Theorem 1.21 shows therefore that 
$${(ab)}^{1/n} = \alpha \beta = a^{1/n} b^{1/n}\,.$$  


\subsection*{1.22 Decimals}
We conclude this section by pointing out the relation between real numbers and decimals. 

Let $x > 0$ be a real. Let $n_0$ be the largest integer such that $n_0 \leq x$. 
(Note that the existence of $n_0$ depends on the archimedean property of $\R$.)
Having chosen $n_0, n_1, \ldots, n_{k-1}\,,$ let $n_k$ be the largest integer such that 
$$ n_0 + \frac{n_1}{10} + \cdots + \frac{n_k}{10^k} \leq x\,.$$
Let $E$ be the set of these numbers\\
 \\
$(5)\quad\quad\quad\quad\quad\quad n_0 + \frac{n_1}{10} + \cdots + \frac{n_k}{10^k} \quad (k = 0, 1, 2, \ldots)\,.$ \\
 \\
Then $x = \text{sup} E$. The decimal expansion of $x$ is\\
 \\
$(6)\quad\quad\quad\quad\quad\quad n_0 \cdot n_1 n_2 n_3 \cdots\;.$\\

Conversely, for any infinite decimal (6) the set $E$ of numbers (5) is bounded above, and (6) is the decimal 
expansion of $\text{sup} E$.

Since we shall never use decimals, we do not enter into a detailed discussion.


\section*{THE EXTENDED REAL NUMBER SYSTEM}

\subsection*{1.23 Definition}
The extended real number system consists of the real field $\R$ and two symbols, $- \infty$ and $+ \infty$. We preserve the original order in $\R$, and define 
$$ - \infty < x < + \infty$$
for every $x \in \R$.

It is then clear that $+ \infty$ is an upper bound of every subset of the extended real number system, and that every nonempty subset has a least upper bound. If, for example, $E$ is a nonempty set of real numbers which is not bounded above in $\R$, 
then $\text{sup} E = + \infty$ in the extended real number system.

Exactly the same remarks apply to lower bounds. 

The extended real number system does not form a field, but it is customary to make the following conventions:

\begin{itemize}
\item[] (a) If $x$ is real then 
$$\quad x + \infty = + \infty\,, \quad x - \infty = - \infty\,,\quad \frac{x}{+ \infty} = \frac{x}{- \infty} = 0\,.$$ 
\item[] (b) If $x > 0$ then $x \cdot (+ \infty) = + \infty\,,\: x \cdot (- \infty) = - \infty\,. $  
\item[] (c) If $x < 0$ then $x \cdot (+ \infty) = - \infty\,,\: x \cdot (- \infty) = + \infty\,. $  
\end{itemize}        

When it is desired to make the distinction between real numbers on the one hand and the symbols $+ \infty$ and $- \infty$ on the other quite explicit, the former are called \tit{finite}.  

\section*{THE COMPLEX FIELD}


\section*{EUCLIDEAN SPACES}


\section*{APPENDIX}


\section*{EXERCISES}





