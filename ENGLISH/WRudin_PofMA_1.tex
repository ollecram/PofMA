\setcounter{chapter}{0}
\renewcommand{\thechapter}{1}
\chapter{The Real and Complex\\ Number Systems}\label{ch:1}
\setcounter{equation}{0}	        % To start with Equation 1
\counterwithin{equation}{chapter}	% Equation numbering will be 2.1 2.2 2.3   ... 

\section*{INTRODUCTION}\label{sec:1.1}

A satisfactory discussion of the main concepts of analysis (such as convergence, continuity, differentiation, and integration) must be based on an accurately defined number concept. We shall not, however, enter into any discussion of the axioms that govern the arithmetic of the integers, but assume familiarity with the rational numbers (i.e., the numbers of the form $m/n$, where $m$ and $n$ are integers and $n \neq 0$).

The rational number system is inadequate for many purposes, both as a field and as an ordered set. (these terms will be defined in Secs. 1.6 and 1.12.) For instance, there is no rational $p$ such that $p^2 = 2$. (we shall prove this presently.) This leads to the introduction of so-called \quotes{irrational numbers} which are often written as infinite decimal expansions and are considered to be \quotes{approximated} by the corresponding finite decimals. Thus the sequence
$$1, 1.4, 1.41, 1.414, 1.4142, \ldots$$ 
\quotes{tends to $\sqrt{2}.$} But unless the irrational number $\sqrt{2}$ has been clearly defined, the question must arise: Just what is it that this sequence \quotes{tends to}?

This sort of question can be answered as soon as the so-called \quotes{real number system} is constructed.
\subsection*{1.1 Example}
We now show that the equation
\begin{align*}
&(1) \quad\quad\quad\quad p^2 = 2
\end{align*}
is not satisfied by any rational $p$. If there were such a $p$, we could write $p = m/n$ where $m$ and $n$ are integers that are not both even. Let us assume this is done. Then (1) implies     
\begin{align*}
&(2) \quad\quad\quad\quad m^2 = 2n^2\,.
\end{align*}
This shows that $m^2$ is even. Hence $m$ is even (if $m$ would be odd, $m^2$ would be odd), and so $m^2$ is divisible by $4$. It follows that the right side of (2) is divisible by $4$, so that $n^2$ is even, which implies that $n$ is even.

The assumption that (1) holds thus leads to the conclusion that both $m$ and $n$ are even, contrary to our choice of $m$ and $n$. Hence (1) is impossible for rational $p$. 

We now examine this situation a little more closely. Let $A$ be the set of all positive rational $p$ such that $p^2 < 2$ and let $B$ consist of all positive rationals $p$ such that $p^2 > 2$. We shall show that $A$ \tit{contains no largest number and} $B$ \tit{contains no smallest.}

More explicitly, for every $p$ in $A$ we can find a rational $q$ in $A$ such that $p < q$, and for every $p$ in $B$ we can find a rational $q$ in $B$ such that $q < p$. 

To do this, we associate with each rational $p > 0$ the number 
\begin{align*}
&(3) \quad\quad\quad\quad q = p - \frac{p^2 - 2}{p + 2} = \frac{2p + 2}{p + 2}\,.
\end{align*}

Then 
\begin{align*}
&(4) \quad\quad\quad\quad q^2 - 2 = \frac{2 (p^2 - 2)}{(p + 2)^2}\,.
\end{align*}

If $p$ is in $A$ then $p^2 - 2 < 0$, (3) shows that $q > p$, and (4) shows that $q^2 < 2$. Thus $q$ is in $A$. 

If $p$ is in $B$ then $p^2 -2 > 0$, (3) shows that $0 < q < p$, and (4) shows that $q^2 > 2$. Thus $q$ is in $B$. 

\subsection*{1.2 Remark}
The purpose of the above discussion has been to show that the rational number system has certain gaps, in spite of the fact that between any two rationals there is another: If $r < s$ then $r < (r + s)/2 < s$. The real number system fills these gaps. This is the principal reason for the fundamental role which it plays in analysis. 

In order to elucidate its structure, as well as that of the complex numbers, we start with a brief discussion of the general concepts of \tit{ordered set} and \tit{field}.

Here is some of the standard set-theoretic terminology that will be used throughout this book. 

\subsection*{1.3 Definitions}
If $A$ is any set (whose elements may be numbers or any other objects), we write $x \in A$ to indicate that $x$ is a member (or an element) of $A$. 

If $x$ is not a member of $A$, we write $x \notin A$.

The set which contains no element will be called the \tit{empty set}. If a set has at least one element, it is called \tit{nonempty}.

If $A$ and $B$ are sets, and every element of $A$ is an element of $B$, we say that $A$ is a subset of $B$, and write $A \subset B$, or $B \supset A$. If, in addition, there is an element of $B$ which is not in $A$, then $A$ is said to be a \tit{proper} subset of $B$. Note that $A \subset A$ for every set $A$.

If $A \subset B$ and $B \subset A$, we write $A = B$. Otherwise, $A \neq B$.   

\subsection*{1.4 Definition}
Throughout Chap. 1, the set of all rational numbers will be denoted by $\Q$. 


\section*{ORDERED SETS}
\subsection*{1.5 Definition}
Let $S$ be a set. An \tit{order} on $S$ is a relation, denoted by $<\;$, with the following two properties:
\begin{enumerate}[(i)]
\item If $x \in S$ and $y \in S$, then one and only one of the statements
$$ x < y, \quad x = y, \quad y < x$$
is true. 
\item If $x, y, z \in S$, if $x < y$ and $y < z$, then $x < z$.
\end{enumerate}

The statement \quotes{$x < y$} may be read as \quotes{$x$ is less than $y$} or \quotes{$x$ is smaller than $y$} or \quotes{$x$ precedes $y$}.

It is often convenient to write $y > x$ in place of $x < y$.

The notation $x \leq y$ indicates that $x < y$ or $x = y$, without specifying which of these two is to hold. In other words, $x \leq y$ is the negation of $x > y$.

\subsection*{1.6 Definition}
An \tit{ordered set} is a set $S$ in which an order is defined.

For example, $\Q$ is an ordered set if $r < s$ is defined to mean that $s - r$ is a positive rational number.

\subsection*{1.7 Definition}
Suppose $S$ is an ordered set, and $E \subset S$. If there exists a $\beta \in S$ such that $x \leq \beta$ for every $x \in E$, we say that $E$ is \tit{bounded above}, and call $\beta$ an \tit{upper bound} of $E$.

Lower bounds are defined in the same way (with $\geq$ in place of $\leq$).  

\subsection*{1.8 Definition}
Suppose $S$ is an ordered set, $E \subset S$, and $E$ is bounded above. Suppose there exists an $\alpha \in S$ with the following properties:
\begin{enumerate}[(i)]
\item $\alpha$ is an upper bound of $E$. 
\item if $\gamma < \alpha$ then $\gamma$ is not an upper bound of $E$. 
\end{enumerate}

Then $\alpha$ is called the \tit{least upper bound} of $E$ [that there is at most one  such $\alpha$ is clear from (ii)] or the \tit{supremum} of $E$, and we write $$\alpha = \text{sup}\,E\,.$$

The \tit{greatest lower bound}, or \tit{infimum}, of a set $E$ which is bounded below is defined in the same manner: The statement $$\alpha = \text{inf}\,E$$
means that $\alpha$ is a lower bound of $E$ and that no $\beta$ with $\beta > \alpha$ is a lower bound of $E$. 
 
\subsection*{1.9 Examples}
\begin{enumerate}[(a)]
\item Consider the sets $A$ and $B$ of Example 1.1 as subsets of the ordered set $\Q$. The set $A$ is bounded above. In fact, the upper bounds of $A$ are exactly the members of $B$. Since $B$ contains no smallest member, $A$ \tit{has no least upper bound in} $\Q$.

Similarly, $B$ is bounded below: The set of all lower bounds of $B$ consists of $A$ and of all $r \in \Q$ with $r \leq 0$. Since $A$ has no largest member, $B$ \tit{has no greatest lower bound in} $\Q$.
   
\item If $\alpha = \text{sup}\,E$ exists, then $\alpha$ may or may not be a member of $E$. For instance, let $E_1$ be the set of all 
$r \in \Q$ with $r < 0$. Let $E_2$ be the set of all $r \in \Q$ with $r \leq 0$. Then 
$$\text{sup}\,E_1 = \text{sup}\,E_2 = 0\,,$$ 
and $0 \notin E_1,\; 0 \in E_2.$

\item Let $E$ consist of all numbers $1/n$, where $n = 1, 2, 3, \ldots$. Then $\text{sup}\,E = 1$, which is in $E$, 
and $\text{inf}\,E = 0$, which is not in $E$.  
\end{enumerate}

\subsection*{1.10 Definition}
An ordered set $S$ is said to have the \tit{least-upper-bound property} if the following is true:
\begin{list}{}{\leftmargin=\parindent\rightmargin=0pt}
\item If $E \subset S$, $E$ is not empty, and $E$ is bounded above, then $\text{sup}\,E$ exists in $S$.\\
Example 1.9(a) shows that $\Q$ does not have the least-upper-bound property. 
\end{list}
We shall now show that there is a close relation between greatest lower bounds and least upper bounds, and that every ordered set with the least-upper-bound property also has the greatest-lower-bound property.   

\subsection*{1.11 Theorem}
Suppose $S$ is an ordered set with the least-upper-bound property, $B \subset S$, $B$ is not empty, and $B$ is bounded below. Let $L$ be the set of all lower bounds of $B$. Then $$\alpha = \text{sup}\,L$$ exists in $S$ and $\alpha = \text{inf}\,B$. In particular $\text{inf}\,B$ exists in $S$.

\begin{list}{}{\leftmargin=\parindent\rightmargin=0pt}
\item \tbf{Proof  } Since $B$ is bounded below, $L$ is not empty. Since $L$ consists of exactly those $y \in S$ which satisfy the inequality $y \leq x$ for every $x \in B$, we see that \tit{every} $x \in B$ \tit{is an upper bound of} $L$. Thus $L$ is bounded above. 
Our hypothesis about $S$ implies therefore that $L$ has a supremum in $S$; call it $\alpha$.

If $\gamma < \alpha$  then (see Definition 1.8) $\gamma$ is not an upper bound of $L$, hence $\gamma \notin B$. It follows that 
$\alpha \leq x$ for every $x \in B$. Thus $\alpha \in L$. 

If $\alpha < \beta$ then $\beta \notin L$, since $\alpha$ is an upper bound of $L$. 

We have shown that $\alpha \in L$ but $\beta \notin L$ if $\beta > \alpha$. In other words, $\alpha$ is a lower bound of $B$, but $\beta$ is not if $\beta > \alpha$. This means that $\alpha = \text{inf}\,B$.       
\end{list}


\section*{FIELDS}


\section*{THE REAL FIELD}


\section*{THE EXTENDED REAL NUMBER SYSTEM}


\section*{THE COMPLEX FIELD}


\section*{EUCLIDEAN SPACES}


\section*{APPENDIX}


\section*{EXERCISES}





